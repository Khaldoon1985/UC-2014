\documentclass[12pt,letterpaper]{article}
\usepackage{slashbox}
\usepackage{pgf}
\usepackage{amsmath,amsthm,amsfonts,amssymb,amscd}
\usepackage{fullpage}
\usepackage{lastpage}
\usepackage{enumerate}
\usepackage{fancyhdr}
\usepackage{mathrsfs}
\usepackage{xcolor}
\usepackage[margin=2cm]{geometry}
\geometry{
 a4paper,
 total={210mm,297mm},
 left=20mm,
 right=20mm,
 top=20mm,
 bottom=20mm,
 }
\setlength{\parindent}{0.0in}
\setlength{\parskip}{0.05in}
\renewcommand{\headrulewidth}{0pt}
\renewcommand{\footrulewidth}{0pt}

% Edit these as appropriate
\newcommand\course{CS6097}
\newcommand\coursename{Wireless Net}
\newcommand\semester{Fall 2014}     % <-- current semester
\newcommand\hwnum{1}                  % <-- homework number
\newcommand\yourname{Chris Park (Kyungmook)} % <-- your name
\newcommand\login{M07068980}           % <-- your NetID
\newcommand\hwdate{Due: Sep 10th 2014 6:00PM} % <-- HW due date
\newcommand\problems{P2.2, P2.3, P2.4, P3.2, P3.4, P3.7}
\newenvironment{answer}[1]{
  \subsubsection*{Problem #1}
}


\pagestyle{fancyplain}
%\setlength{\headheight}{72pt}
%\chead{\textbf{\Large{Homework \hwnum}}}
%\rhead{\yourname\ \login\\\course\coursename\ --- \semester\\\hwdate\\\problems\\}
%\headsep 50pt


\begin{document}

\begin{flushright}
\yourname\ \login\\\course\coursename\ --- \semester\\\hwdate\\\problems\\
\end{flushright}

\begin{center}
\textbf{\Large{Homework \hwnum}}
\end{center}

\begin{answer}{1}

\textbf{{P2.2: } A random digit generator on a computer is activated three times consecutively to
simulate a random three-digit number.}

\begin{enumerate}[(a)]
\item
\textbf{How many random three-digit numbers are possible?\\}
If we include 0 as the first digit,\\
10 * 10 * 10 = \pgfmathparse{int(10*10*10)}\pgfmathresult\\
\pgfmathparse{int(10*10*10)}\pgfmathresult numbers are possible.

\item
\textbf{How many numbers will begin with the digit 2?\\}
First digit is fixed; therefore, 1 * 10 * 10 = \pgfmathparse{int(1*10*10)}\pgfmathresult\\
\pgfmathparse{int(1*10*10)}\pgfmathresult\\ numbers will begin with the digit 2.

\item
\textbf{How many numbers will end with the digit 9?\\}
Last digit is fixed; therefore, 10 * 10 * 1 = \pgfmathparse{int(10*10*1)}\pgfmathresult\\
\pgfmathparse{int(10*10*1)}\pgfmathresult\\ numbers will end with the digit 9.

\item
\textbf{How many numbers will begin with the digit 2 and end with the digit 9?\\}
The first and the last digits are fixed; therefore, 1 * 10 * 1 = \pgfmathparse{int(1*10*1)}\pgfmathresult\\
\pgfmathparse{int(1*10*1)}\pgfmathresult\\ numbers will begin with the digit 2 and end with the digit 9.

\item
\textbf{What is the probability that a randomly formed number ends with 9 given that
it begins with a 2?\\}
Given the number begins with the digit 2, 100 numbers will be formed.\\
Out of the 100 numbers, 10 numbers will end with the digit 9.\\
Therefore, the probability that a randomly formed number ends with 9 given that it begins with a 2 is 10/100 ~= \pgfmathparse{int(10)}\pgfmathresult\%\\

\end{enumerate}

\end{answer}

\begin{answer}{2}

\textbf{{P2.3: } A snapshot of the traffic pattern in a cell with 10 users of a wireless system is given
as follows:}

\begin{table}[htbp]
   \centering
   \begin{tabular}{|c|c|c|c|c|c|c|c|c|c|c|c|}\hline % Column formatting
         User Number&1&2&3&4&5&6&7&8&9&10\\\hline
      Call Initiation Time&0&2&0&3&1&7&4&2&5&1\\\hline
      Call Holding Time&5&7&4&8&6&2&1&4&3&2\\\hline
   \end{tabular}
   \label{tab:booktabs}
\end{table}

\begin{enumerate}[(a)]
\item
\textbf{Assuming the call setup/connection and call disconnection time to be zero, what
is the average duration of a call?}

Let\\
$t_{hold}$ = call holding time\\
$n_{users}$ = number of users = 10\\
$t_{call}$ = average call duration\\

$t_{call}$ = $\frac{\sum{t_{hold}}}{n_{users}}$ = $\frac{5+7+4+8+6+2+1+4+3+2}{10}$ = \pgfmathparse{(5+7+4+8+6+2+1+4+3+2)/10}\pgfmathresult per customer\\

\item
\textbf{What is the minimum number of channels required to support this sequence of
calls?}

O indicates channel being used
X indicates channel being initiated
E indicates neither
\begin{table}[htbp]
   \centering
   \begin{tabular}{|c|c|c|c|c|c|c|c|c|c|c|c|}\hline % Column formatting
        \backslashbox{Time}{Customer}&1&2&3&4&5&6&7&8&9&10\\\hline
				1&O&X&O&X&X&X&X&X&X&X\\\hline
				2&O&X&O&X&O&X&X&X&X&O\\\hline
				3&O&O&O&X&O&X&X&O&X&O\\\hline
				4&O&O&O&O&O&X&X&O&X&E\\\hline
				5&O&O&E&O&O&X&O&O&X&E\\\hline
				6&E&O&E&O&O&X&E&O&O&E\\\hline
				7&E&O&E&O&O&O&E&E&O&E\\\hline
				8&E&O&E&O&E&O&E&E&O&E\\\hline
				9&E&E&E&O&E&E&E&E&E&E\\\hline
				10&E&E&E&O&E&E&E&E&E&E\\\hline
   \end{tabular}
   \label{tab:booktabs}
\end{table}


\item
Show the allocation of channels to different users for part (b) of this problem.
\item
Given the number of channels obtained in part (b), for what fraction of time
are the channels utilized?
\end{enumerate}

\end{answer}

\begin{answer}{3}

{P2.4: } A department survey found that four of ten graduate students use CDMA cell
phone service. If three graduate students are selected at random, what is the
probability that the three graduate students use CDMA cell phones?

\end{answer}

\begin{answer}{4}

{P3.2: } Consider an antenna transmitting a power of 5W at 900 MHz. Calculate the
received power at a distance of 2km if propagation is taking place in free space.
 
\end{answer}

\begin{answer}{5}

{P3.4: } The transmission power is 40W, under a free-space propagation model,

\begin{enumerate}[(a)]
\item
What is the transmission power in unit of dBm?
\item
The receiver is in a distance of 1000 m; what is the received power, assuming
that the carrier frequency fc = 900 MHz and Gt = Gr = 0 dB?
\item
Express the free space path loss in dB.
\end{enumerate}


\end{answer}

\begin{answer}{6}

{P3.7: } Consider an antenna transmitting at 900 MHz. The receiver is traveling at a speed
of 40 km/h. Calculate its Doppler shift.

\end{answer}

\end{document}
