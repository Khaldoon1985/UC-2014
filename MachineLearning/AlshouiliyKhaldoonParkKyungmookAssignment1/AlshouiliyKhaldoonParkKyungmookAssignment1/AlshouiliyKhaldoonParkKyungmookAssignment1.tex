\documentclass[12pt,letterpaper]{article}
\usepackage{pgf}
\usepackage{amsmath,amsthm,amsfonts,amssymb,amscd}
\usepackage{fullpage}
\usepackage{lastpage}
\usepackage{enumerate}
\usepackage{fancyhdr}
\usepackage{mathrsfs}
\usepackage{xcolor}
\usepackage[margin=3cm]{geometry}
\setlength{\parindent}{0.0in}
\setlength{\parskip}{0.05in}

% Edit these as appropriate
\newcommand\course{20CS6037}
\newcommand\semester{Fall 2014}     % <-- current semester
\newcommand\hwnum{1}                  % <-- homework number
\newcommand\yourname{FirstName LastName } % <-- your name
\newcommand\login{Mnumber}           % <-- your NetID
\newcommand\hwdate{Due: Enter Due date}           % <-- HW due date

\newenvironment{answer}[1]{
  \subsubsection*{Problem #1}
}


\pagestyle{fancyplain}
\headheight 35pt
\lhead{\yourname\ \login\\\course\ --- \semester}
\chead{\textbf{\Large Homework \hwnum}}
\rhead{\hwdate}
\headsep 10pt

\begin{document}

\noindent \emph{Homework Notes:} Add information here on your study group, number of hours you spent on the homework, and other relevant information.

\begin{answer}{1}

You can write aligned equations as follows:
\begin{eqnarray*}
a &\sim& p(a)\\
b &\sim& p(b).
\end{eqnarray*}

\textbf{(7 points)} Pairwise independence does not imply mutual independence.  Two random variables, $X_i$, i=1,2 are independent if\\
P($X_i$ $|$ $X_j$) = P($X_i$), for i,j=1,2, i $\neq$ j\\
and therefore\\

P($X_i$, $X_j$) = P($X_j$)\\ P($X_i$ $|$ $X_j$) = P($X_i$)P($X_j$)\\

Now, given n random variables, we say that there are mutually independent if\\
P($X_i$| $X_S$) = P($X_i$) for all subsets S of \{1, 2, $\cdots$ , n\}which do not contain i, and therefore\\ P($X_1$,  $\cdots$ , $X_n$) = xP($X_1$) $\cdots$ xP($X_n$)\\


You can write inline equations: $a \sim p(b)$, or one line equations:
\[
a \sim p(a).
\]

\begin{enumerate}[(1)]
\item
\textbf{Show that pairwise independence between all pairs of variables ($X_i$, $X_j$), does NOT imply mutual independence. Note: it is enough to give an example.}

SUPPOSE A BOX CONTAINS 4 TICKETS LABELLED BY
331	323	233	333
LET US CHOOSE ONE TICKET AT RANDOM, AND CONSIDER THE RANDOM EVENTS
A1={1 OCCURS AT THE FIRST PLACE}
A2={1 OCCURS AT THE SECOND PLACE}
A3={1 OCCURS AT THE THIRD PLACE}
P(A1)=1/2	P(A2)=1/2	P(A3)=1/2

A1A2={112}	A1A3={121}	A2A3={211}
P(A1A2)=P(A1A3)=P(A2A3)=1/4.
So we conclude that the three events A1, A2, A3 are pairwise independent.
However
A1A2A3=f
P(A1A2A3)=0¹P(A1)P(A2)P(A3)=(1/2)3


\item
\textbf{Show mutual independence implies pairwise independence.}

For example, for four events A, B, C, D to be mutually independent, we must have
 	P(A∩B∩C∩D) = P(A)P(B)P(C)P(D),
P(A∩B∩C) = P(A)P(B)P(C),
P(A∩B∩D) = P(A)P(B)P(D),
P(A∩C∩D) = P(A)P(C)P(D),
P(B∩C∩D) = P(B)P(C)P(D), 
P(A∩B) = P(A)P(B), P(A∩C) = P(A)P(C), P(A∩D) = P(A)P(D),
P(B∩C) = P(B)P(C), P(B∩D) = P(B)P(D), P(C∩D) = P(C)P(D).

\end{enumerate}

\end{answer}

\begin{answer}{2} 
\textbf{(8 points)} Let X and Y be two discrete random variables which are identically distributed but not necessarily independent.\\
Define\\
R = 1 – H(Y$|$X) / H(X)

\begin{enumerate}[(a)]
\item
\textbf{Show that R = I(X,Y) / H(X)}

R = 1 - H(Y|X)/H(X) = ( H(X)-H(Y|X) )/ H(X)
I(X,Y) / H(X) = 1 - H(X|Y) / H(X)
%Identically distributed therefore
H(Y|X) = H(X|Y)
H(X) = H(Y|X)

\item
\textbf{Show that 0 $<=$ R $<=$ 1}
\item
\textbf{When is R = 0?}
\item
\textbf{When is R = 1?}
\end{enumerate}


\end{answer}


\end{document}
